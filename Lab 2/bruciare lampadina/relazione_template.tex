\documentclass[a4paper]{article}
\usepackage[utf8]{inputenc}
\usepackage{graphicx}

\begin{document}
\section{introduzione}
\begin{enumerate}
   \item quello che fai durante la relazione
   \item riassunto sintetico degli strumenti
   \item scrivere se dati dipendono fortemente dallo strumento (marca, modello, etc)
\end{enumerate}

\section{dati}
metti dati se non sono troppi, 6/7 righe con 3/4 colonne massimo

\section{Risultati}
\begin{enumerate}
   \item risultati sotto forma grafico, istogramma, etc
   \item deve illustrare fenomeno fisico spiegato e deve essere immediatamente leggibile e capibile
   \item se ha pochi punti scrivi barre d'errori, se invece ha tanti punti non si mettono su ogni punto, in questi casi, si mette ogni 3/4 punti. 
   \item si deve mettere qualche commento per ciascun grafico
   \item scrivi forse se ci sono errori sistematici (vanno individuati in caso se ci sono)
   \item \textbf{NON DIRE SE È USCITO BENE L'ESPERIENZA MI RACCOMANDO!!!!!}
\end{enumerate}

\section{Passaggi}
\begin{enumerate}
   \item preso una lampadina e calcolato resistenza necessaria con excel per vedere che resistenza scegliere per regolare il voltaggio
   \item montato circuito
   \item prese le misure aumentando di circa 0,3 A, poi di 0,5A dall'alimentatore
   \item verificato il valore della resistenza
\end{enumerate}

misurato corrente perchè è misura diretta, invece di voltometro

\end{document}