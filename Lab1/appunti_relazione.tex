
\documentclass[a4paper]{article}
\usepackage[utf8]{inputenc}
\usepackage{graphicx} % Required for inserting images
\usepackage[italian]{babel}
\usepackage{gensymb}
\usepackage{array}
\usepackage{amsmath}
\usepackage{hyperref}
\usepackage{siunitx}
\usepackage{caption}
\usepackage{systeme}

\makeindex
\setlength{\parindent}{0pt}

\title{Modulo di Young e di Coulomb}
\author{Francesco Giuliano Rossi}
\date{}

\begin{document}
\maketitle
\tableofcontents

\section{Modulo di Young}
-------------------------------------------------------------------------- \\
 Per misurare il modulo di Young si usa il principio della leva ottica. Il filo sarà attaccato al blocco, e avvolto attorno un cilindro, e poi attaccato a una massa. Sul cilindro è uno specchio e c'è una sorgente luminosa dall'altra parte. Se metto masse sul piattello, il filo si allunga, ruotando sull'asse e gira il cilindro di un certo angolo $\theta$, allora la sorgente luminosa anche si sposterà. Questa variazione di angolo si potrà misurare vedendo il cambiamento su una parete. 

$\Delta L$ sarà una funzione di h $\Delta L = f(h)$. Per farlo nel modo più semplice possibile, vogliamo che l'altezza della lampada e l'altezza dello specchio sono alla stessa altezza all'inizio. Quando ruoto il cilindro, ruota anche lo specchio. 
Quando io ruoto lo specchio, rispetto alla normale, il raggio riflesso avrà un angolo $\alpha$. 
Per angoli e parte di ottica guarda ipad Teoria LDF. Risultato è che scale factor è di 200. Per l'errore, guardiamo $\Delta[\Delta L] = \frac{R_c}{2d} \Delta h]$ e per la propagazione dell'errore 
\begin{equation*}
    \frac{\Delta [\Delta L]}{\Delta l} = \frac{\Delta R_c}{R} + \frac{\Delta d}{d} + \frac{\Delta h}{h}
\end{equation*}
e si fa un errore su h di circa $0.5*10^-4$, e l'errore su $\Delta L = \frac{1}{200} \Delta h ~ 5*10^-6$ 

Con il setup che abbiamo, ritornando alla equazione per $\sigma$ si ha che 
\begin{equation*}
    \systeme{
        \Delta L = R_c \alpha = R_c 1/2 tan^-1 \frac{h}{d},
        F = Mg,
        S_F = \pi (D-F/2)^2
    }
\end{equation*}
e ci permette di ricavare il modulo di Young E in funzione di quantità misurabili 
\begin{equation}
    E = \frac{8gL}{\pi R_c D^2_f} \frac{M}{tan^-1 \frac{h}{d}} 
\end{equation}
e per piccoli angoli si può usare l'approssimazione $tan \alpha ~ \alpha$. Tuttavia, Tutto il mio filo si allunga, ma non tutto ruota il cilindro. La lunghezza di $L = L_t + \pi R_c$, dove $L_t$ è la parte del filo che tocca il cilindro e l'altra componente è quella che fa girare il cilindro. Per il diametro del filo si usano vari punti diversi e si prende la media vogliamo che filo non scorre sul cilindro, quindi dobbiamo verificare ogni volta di non avere deformazione o scorrimento ogni volta. Setup iniziale con piattello e poi qualche massa per avere tensione 150grammi messi in più (perché misuriamo allungamenti relativi). Una rotazione di 1 mm equivale a 20 mm sulla scala.
Nella stima di $\frac{\Delta E}{E}$ non è diverso usare l'andamento semplificato rispetto alla formula completa perché ho 1 o 2 max cifre significative nella scrittura dell'errore, è uguale usare le formule 
\begin{equation}
    \frac{Md}{h}  \text{si può anche usare} \space \frac{M}{\arctan \frac{h}{d}}
\end{equation}
fino a quando l'approssimazione è valida?

se prendiamo 
\begin{equation}
    |\frac{\tan \alpha - \alpha}{\alpha}| \text{e sappiamo che} E \propto \frac{1}{\arctan \frac{h}{d} \text{e che} \Delta L = R_c \alpha} \implies \alpha = \frac{1}{2} \arctan \frac{h}{d}
\end{equation}
sviluppando in serie di taylor, ci risulta che 
\begin{equation}
    |\tan \alpha| = |\frac{\alpha^3}{3}| \text{ricordando che} \alpha = \frac{h}{d} \implies = |\frac{\frac{h}{d}^2}{3}| < \frac{\Delta h}{h}
\end{equation}
Dobbiamo avere che $h^3 < 3d\Delta h \implies h < \sqrt[3]{3d\Delta h}$ e visto che d ~ 1 m e $\Delta h ~ 10^{-3}$, e quindi $h<\frac{\sqrt[3]{3}}{10}$ e quindi se h è maggiore di 14 cm non possiamo usare l'approssimazione $\tan \alpha ~ \alpha$. 


Visto che il modulo di Young è cmposto da due parti:
\begin{equation}
    E_i = \frac{8gLd}{\pi R_C D^2_f} * \frac{M_i}{h_i}
\end{equation}
dove $A = \frac{8gLd}{\pi R_C D^2_f}$ e $ B = \frac{M_i}{h_i}$

\section{Modulo di Coulomb}
------Prendiamo un cavo di acciaio e provochiamo una torsione, quindi applichiamo un certo momento $M$ con un angolo di rotazione $\theta$. Questa rotazione lo possiamo vedere come una rotazione di angolo $\theta$ nella faccia più bassa rispetto a quella più alta. Se si apre il cilindro, si ottiene un parallelepipedo, che ha circonferenza come base e lunghezza come altezza. Dopo la torsione, si vede che la faccia inferiore ha un angolazione $\theta$ rispetto a quella più sopra e questo $\phi$ corrisponde a un certo $r\theta$. Così si può usare la relazione $\frac{\Delta L}{L} = \frac{r\theta}{L}$. Quindi la torsione può essere visto come una applicazione di sforzi di taglio che ruotano un oggetto. \\
\\
Quando applichiamo le forze, si giunge una posizione di equilibrio, e visto che il materiale ha proprietà elastiche, ha una forza di richiamo. Le coppie di forze sono applicate sulla base inferiore. 

Per il cilindro, si ha che 
\begin{equation}
    \sigma = \frac{dF}{2\pi r dr} = G \frac{\Delta L}{L} = G \frac{r\theta}{L} \implies dF \ G \frac{2\pi r^2 \theta}{L} dr
\end{equation}
Per ottenere il momento della forza $dM = r x dF$ integriamo su tutto il raggio del filo e si ottiene che $dM = \frac{\pi G}{2} \frac{R^4}{L} \theta$ da cui definiamo $k = G \frac{\pi r^4}{2L}$ e si introduce il modulo di Coulomb $\mu = \frac{\pi G}{2} \implies M = \mu \frac{R^4}{L}\theta$. Dopo che si toglie il momento esterno, il cavo inizia a oscillare con moto armonico, quindi si deve misurare il periodo del moto armonico.

Per il momento torcente: $\mu = 4\pi^2 \frac{LI}{R^4}\frac{1}{T^2}$ che passa per il centro del indicatore attaccato al filo di acciaio


Immaginiamo di avere due corone circolari che hanno entrambi diametro esterno, interno, altezza, e massa. Vogliamo calcolare il momento d'inerzia rispetto all'asse di simmetria (quello passante per il centro). Per definizione di momento d'inerzia $I = \int_{\Delta m} r^2_{perpendicolare} dm$. Inoltre, per definizione si ha che $dm = \rho dV$ e $dV = h 2\pi r dr$ quindi mettendo tutto insieme si ha che 
\begin{equation}
    I \ \rho H 2 \pi \int_{R_i}^{R_e} r^3 dr = \frac{\rho H \pi}{2} (R^4_e - R^4_i)
\end{equation}
sappiamo che $\rho = \frac{M}{V} = \frac{M}{\pi (R^2_e - R^2_i)H}$ e allora sostituendo nell'equazione prima si ha che:
\begin{equation*}
    \systeme{
    I = \frac{M}{\pi (R^2_e - R^2_i)H}  \frac{H\pi}{2} (R_e^2-R^2_i)(R_e^2+R^2_i),
    I = \frac{M}{2} (R_e^2+R^2_i)
}
\end{equation*}

Il problema è che il cavo di acciaio non è perfettamente circolare, e quindi il momento risulterebbe
\begin{equation}
    T^2 = 4\pi^2 \frac{IL}{\mu R^4} \implies T^2 = 4\pi^2 \frac{L}{\mu R^4} (I_i + I_p + I_{ef})
\end{equation}
e per risolvere questo problema, facciamo come il pendolo dove calcoliamo I per varie masse, per quando è scarico, e poi facciamo la differenza così $\mu$ risulta $\mu 4\pi^2 \frac{I_i L}{R^4_f (I^2_i - T^2_{i,0})}$ ------
\end{document}