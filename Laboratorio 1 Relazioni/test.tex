\documentclass[a4paper]{article}
\usepackage[utf8]{inputenc}
\usepackage{graphicx} % Required for inserting images
\usepackage[italian]{babel}
\usepackage{gensymb}
\usepackage{array}
\usepackage{amsmath}
\usepackage{hyperref}
\usepackage{siunitx}
\usepackage{caption}
\usepackage{systeme}
\usepackage{listings}
\usepackage{color}

\definecolor{dkgreen}{rgb}{0.2,0.48,0.32}
\definecolor{gray}{rgb}{0.5,0.5,0.5}
\definecolor{mauve}{rgb}{0.58,0,0.82}

\lstset{
  basicstyle={\small\ttfamily},        % the size of the fonts that are used for the code
  breakatwhitespace=false,         % sets if automatic breaks should only happen at whitespace
  breaklines=false,                 % sets automatic line breaking
  captionpos=b,                    % sets the caption-position to bottom
  commentstyle=\color{dkgreen},    % comment style
  extendedchars=true,              % lets you use non-ASCII characters; for 8-bits encodings only, does not work with UTF-8
  keepspaces=true,                 % keeps spaces in text, useful for keeping indentation of code (possibly needs columns=flexible)
  keywordstyle=\color{blue},       % keyword style
  language=[95]Fortran,                 % the language of the code
  numbers=left,                    % where to put the line-numbers; possible values are (none, left, right)
  numbersep=5pt,                   % how far the line-numbers are from the code
  numberstyle=\tiny\color{gray}, % the style that is used for the line-numbers
  rulecolor=\color{black},         % if not set, the frame-color may be changed on line-breaks within not-black text (e.g. comments (green here))
  showspaces=false,                % show spaces everywhere adding particular underscores; it overrides 'showstringspaces'
  showstringspaces=false,          % underline spaces within strings only
  showtabs=false,                  % show tabs within strings adding particular underscores
  stepnumber=1,                    % the step between two line-numbers. If it's 1, each line will be numbered
  stringstyle=\color{mauve},     % string literal style
  tabsize=4,                       % sets default tabsize to 2 spaces
  title=\lstname                   % show the filename of files
}

\begin{document}
\begin{lstlisting}[firstnumber=1]
  program harm
 implicit none
! Grafico fatto con dt=0.01, n.step = 10000, massa = 4, kappa = 7, pos(0)/0, vel(0) = 1


 real    :: massa=1.0,kappa=1.0 ! valori di default per massa e cost. elastica
 real    :: dt,ekin,epot
 real    :: pos,vel,vel_parziale,f, omega, pos0, vel0, pos_analitica, velocita
 integer :: nstep,it
 write(unit=*,fmt="(a)",advance="no")"delta t : "   ! il formato (a) chiede che il dato sia trattato come
 read*,dt                                           ! caratteri e advance="no"  sopprime il
 write(unit=*,fmt="(a)",advance="no")"n.step: "     ! carattere "a capo"  alla fine della linea per cui
 read*,nstep                                        ! la prossima operazione di lettura/scrittura inzia
 write(unit=*,fmt="(a)",advance="no")"massa: "      ! sulla stessa riga di schermo di quella corrente
 read*,massa
 write(unit=*,fmt="(a)",advance="no")"kappa: "
 read*,kappa
 write(unit=*,fmt="(a)",advance="no")"pos(0): "
 read*,pos
 write(unit=*,fmt="(a)",advance="no")"vel(0): "
 read*,vel
pos0=pos
vel0=vel
omega=sqrt(kappa/massa)

 it=0        ! step 0 : valori iniziali

 write(unit=1,fmt=*)it,it*dt,pos,vel
 epot =  0.5 * kappa * pos**2
 f    = - kappa * pos
 ekin =  0.5 * massa * vel**2
 write(unit=2,fmt=*)it,dt*it,ekin,epot,ekin+epot

 do it = 1,nstep
    pos = pos + vel * dt + 0.5* f/massa * dt**2
    vel_parziale = vel + 0.5 * dt * f/massa         !  prima parte della formula per le velocita'
    f    = - kappa * pos
    epot =  0.5 * kappa * pos**2
    vel = vel_parziale + 0.5 * dt * f/massa         !  la formula per le velocita' viene completata qui
    pos_analitica=pos0*cos(omega*it*dt)+(vel0/omega)*sin(omega*it*dt)
    velocita = -omega*pos0*sin(omega*it*dt)+vel0*cos(omega*it*dt)
    write(unit=1,fmt=*)it,it*dt,pos,vel, pos_analitica, velocita
    ekin = 0.5 * massa * vel**2
    write(unit=2,fmt=*)it,it*dt,ekin,epot,ekin+epot
    write(unit=3, fmt=*) pos, pos_analitica, abs(pos-pos_analitica), vel, velocita, abs(velocita-vel) !scrive posizione, posizione aalitica, differenza tra i due, velocita formula, velocita analitica, differenza tra i due

 end do
 end program harm

\end{lstlisting}
\end{document}